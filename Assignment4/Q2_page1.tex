%Document class to define the type of document
\documentclass[12pt]{article}

%amsmath package is used for align 
\usepackage{amsmath}

%Used to refer and change the default settings of a link
\usepackage{hyperref}
\hypersetup{
colorlinks=true,
linkcolor=blue
}
\title{Hello World!}
\author{Achintya Gupta}
\date{1 November 2022}

%Necessary to wrap all content in document
\begin{document}

\maketitle

%Starting first section
\section{Getting Started}
\textbf{Hello World!} Today I am learning \LaTeX. \LaTeX\ is a great platform for writing math. I can write inline math such as $a^2 + b^2 = c^2$. I can also give equations their own space: 

%Writes equation in new line and also adds the tag (1) at the end of the line
\begin{equation}
  \gamma^2 + \theta^2 = \omega^2  
\end{equation}

"Maxwell's equations" are named for James Clarke Maxwell and are as follow:

%Align environment to align equations by = sign and text by the first letter of text
%Also added labels and tag for refering to them later
\begin{align*}
\vec\nabla\cdot\vec{E} &= \frac{\rho}{\epsilon_0} & &\textrm{Gauss's Law}\label{eq:2}\tag{2}  \\
\vec\nabla\cdot\vec{B} &= 0& &\textrm{Gauss's Law for Magnetism}\label{eq:3}\tag{3}\\
\vec\nabla \times \vec{E} &= -\frac{\partial\vec{B}}{\partial t} & &\textrm{Faraday's Law of Induction}\label{eq:4}\tag{4}\\
\vec\nabla \times \vec{B} &= \mu_0\left( \epsilon_0\frac{\partial\vec{E}}{\partial t} + \vec{J} \right) & &\textrm{Ampere's Circuital law}\label{eq:5}\tag{5}
\end{align*}

Equations \ref{eq:2}, \ref{eq:3}, \ref{eq:4}, \ref{eq:5} are some of the most important equations in Physics.

\section{What about Matrix Equations?}
%Equation* removes the numbering of equation
\begin{equation*}
%pmatrix makes a matrix within parentheses
\begin{pmatrix}
a_{11} & a_{12} & \cdots & a_{1n} \\
a_{21} & a_{22} & \cdots & a_{2n} \\
\vdots  & \vdots  & \ddots & \vdots  \\
a_{n1} & a_{n2} & \cdots & a_{nn} 
\end{pmatrix}
%bmatrix makes a matrix within brakcets
\begin{bmatrix}
v_1 \\
v_2 \\
\vdots \\
v_n
\end{bmatrix}
=
%matrix without brackets
\begin{matrix}
w_1 \\
w_2 \\
\vdots \\
w_n
\end{matrix}
\end{equation*}


\end{document}
